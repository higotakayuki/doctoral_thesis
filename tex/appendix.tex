\chapter{Comparison between EDA and the EAPM (CE)}
\label{exp-edace}
This appendix provides experimental studies on comparisons
between EDA and the EAPM.
In this appendix, the EAPM is denoted by CE.
In spite of the theoretical aspects of CE,
CE does not work well in practice.

\section{Experimental Setup}
\subsection{EDA Setting}
We employ UMDA
\cite{muhlenbein:umda} as the EDA.
Thus, the probability model is defined as
\begin{equation}
 p(x|w)=\prod_{i=0}^{i=d-1} p(x_i|w_i)
\end{equation}
and ML estimation is employed for building the probability models.
Here, the learning rate $\alpha$ is introduced.
The parameter $w$ is updated by the following equation:
\begin{equation}
 w_{new}=(1-\alpha) w_{old} + \alpha w_{ML},
\end{equation}
where $w_{new},w_{old},w_{ML}$ are the new parameter,
previous parameter, and ML estimator, respectively.
This mechanism affords stable estimation.

The selection operator employed is the truncation selection operator.
The truncation selection operator includes the cutoff rate parameter $c$,
which represents the percentage of samples that are removed.
For example, if $c=0.3$ and
the number of generated samples is $100$,
then the best $70=100 \times (1-0.3)$ samples are selected
and the rest are discarded.
All the parameter settings are described as follows:
\begin{itemize}
 \item The number of generated samples in one sampling $M$: $100$, $500$,
$1000$, $3000$, or $6000$.
 \item Cutoff rate $c$: $0.1$, $0.3$, or $0.5$.
 \item Learning rate $\alpha$: $0.5$.
\end{itemize}
These values are experimentally determined.


\subsection{CE Setting}
CE uses the same probability model and estimation method
as EDA.
However, instead of truncation selection,
CE employs the $(1-\delta)$-quantile method \cite{rubinstein:ce},
which selects the best $k=M\times\delta$ samples, where $M$ is
the number of generated samples, and removes the rest.
Truncation selection and the \mbox{$(1-\delta)$-quantile} method 
are basically the same:
the parameter $(1-\delta)$ corresponds to
the cutoff rate in truncation selection. 
Thus, $(1-\delta)$ is referred to as the cutoff parameter in this paper.
All the parameter settings are described as follows:
\begin{itemize}
 \item The number of generated samples in one sampling $M$: $100$, $500$,
$1000$, $3000$, or $6000$.
 \item Cutoff rate $c=1-\delta$: $0.3$, $0.5$, or $0.7$.
 \item Learning rate $\alpha$: $0.5$.
\end{itemize}
These values are experimentally determined.

\section{Results}

\begin{table}[tbp]
%\footnotesize
\centering
\caption{\tabletitle{EDA}{Onemax}}
%\vskip 0.05in
\begin{tabular}{|r|r|rr|rr|}
\hline
\multicolumn{1}{|c|}{Samples} & \multicolumn{1}{c|}{Cutoff} &
 \multicolumn{2}{c|}{Best} & 
 \multicolumn{2}{c|}{Evaluations}  \\ \hline
100 & 0.5 & --400 &  (0) & 8570 & (148.66) \\ \hline
500 & 0.5 & --400 &  (0) & 41950 &  (610.33) \\ \hline
500 & 0.3 & --400 &  (0) & 64750 &  (512.35) \\ \hline
1000 & 0.5 & --400 &  (0) & 85200 &  (1326.65) \\ \hline
1000 & 0.3 & --400 &  (0) & 130200 &  (1400) \\ \hline
500 & 0.1 & --400 &  (0) & 159600 &  (2406.24) \\ \hline
3000 & 0.5 & --400 &  (0) & 260400 &  (3231.1) \\ \hline
1000 & 0.1 & --400 &  (0) & 314900 &  (4109.74) \\ \hline
3000 & 0.3 & --400 &  (0) & 391800 &  (4069.4) \\ \hline
6000 & 0.5 & --400 &  (0) & 523800 &  (7613.15) \\ \hline
6000 & 0.3 & --400 &  (0) & 792000 &  (8485.28) \\ \hline
3000 & 0.1 & --400 &  (0) & 945600 &  (5969.92) \\ \hline
6000 & 0.1 & --400 &  (0) & 1891200 &  (6462.2) \\ \hline
100 & 0.3 & --399.9 &  (0.3) & 13400 &  (275.68) \\ \hline
100 & 0.1 & --395.1 &  (1.7) & 36030 &  (1500.03) \\ \hline
\end{tabular}
\label{ann-onemax}
\end{table}

\begin{table}[tbp]
\centering
\caption{\tabletitle{EDA}{1D Ising}}
\begin{tabular}{|r|r|rr|rr|}
\hline
\multicolumn{1}{|c|}{Samples} & \multicolumn{1}{c|}{Cutoff} &
 \multicolumn{2}{c|}{Best} & 
 \multicolumn{2}{c|}{Evaluations}  \\ \hline
3000 & 0.3 & --364.8 &  (4.02) & 1114800&  (62183.29)\\  \hline
3000 & 0.5 & --364.4 &  (2.94) & 788400 &  (54212.91) \\  \hline
1000 & 0.5 & --363.8 &  (6.54) & 228600 &  (22037.24) \\  \hline
6000 & 0.5 & --363.6 &  (5.78) & 1839600 &  (120365.44) \\   \hline
6000 & 0.3 & --362.6 &  (4.2) & 2463000 &  (118922.66) \\  \hline
3000 & 0.1 & --360.8 &  (3.82) & 2260200 &  (49060.78) \\  \hline
1000 & 0.3 & --359.8 &  (4.33) & 307800 &  (12064.82) \\  \hline
500 & 0.3 & --358.4 &  (4.96) & 146900 &  (13931.62) \\  \hline
500 & 0.5 & --358 &  (3.9) & 95700 &  (4648.66) \\  \hline
1000 & 0.1 & --356.8 &  (4.21) & 663200 &  (35312.32) \\  \hline
100 & 0.5 & --354 &  (6.69) & 13720 &  (570.61) \\  \hline
500 & 0.1 & --352.8 &  (5.38) & 308550 &  (22005.06) \\  \hline
100 & 0.3 & --348.6 &  (4.39) & 20510 &  (1328.5) \\  \hline
100 & 0.1 & --338.2 &  (5.55) & 45170 &  (6008.5) \\  \hline
6000 & 0.1 & --322.4 &  (5.35) & 2904000 &  (0) \\  \hline
\end{tabular}
\label{ann-1d-ising}
%\vskip 0.05in
\end{table}

\begin{table}[tbp]
\centering
\caption{\tabletitle{EDA}{2D Ising}}
\begin{tabular}{|r|r|rr|rr|}
\hline
\multicolumn{1}{|c|}{Samples} & \multicolumn{1}{c|}{Cutoff} &
 \multicolumn{2}{c|}{Best} & 
 \multicolumn{2}{c|}{Evaluations}  \\ \hline
3000 & 0.5 & --719 &  (15.68) & 746700 & (63617.69) \\ \hline
6000 & 0.3 & --714 &  (18.57) & 2269800 &  (277094.14) \\ \hline
3000 & 0.3 & --709.6 &  (8.04) & 1073400 &  (130579.63) \\ \hline
1000 & 0.5 & --706.6 &  (12.33) & 213100 &  (22997.61) \\ \hline
6000 & 0.5 & --705.4 &  (12.84) & 1671000 &  (165043.63) \\ \hline
1000 & 0.3 & --705 &  (8.06) & 321600 &  (38257.55) \\ \hline
3000 & 0.1 & --698.6 &  (15.07) & 2625900 &  (261206.99) \\ \hline
500 & 0.5 & --697 &  (13.89) & 94800 &  (6021.63) \\ \hline
500 & 0.3 & --694.8 &  (9.39) & 151400 &  (14902.68) \\ \hline
500 & 0.1 & --688.6 &  (17.32) & 370050 &  (56346.45) \\ \hline
1000 & 0.1 & --686 &  (9.34) & 807500 &  (118196.66) \\ \hline
100 & 0.5 & --680.8 &  (10.59) & 14230 &  (445.08) \\ \hline
100 & 0.3 & --664.4 &  (14.31) & 22410 &  (1602.78) \\ \hline
6000 & 0.1 & --649.8 &  (12.79) & 2904000 &  (0) \\ \hline
100 & 0.1 & --632.2 &  (12.47) & 47430 &  (2609.23) \\ \hline
\end{tabular}
\label{ann-2d-ising}
\end{table}

\begin{table}[tbp]
%\footnotesize
\centering
\caption{\tabletitle{CE}{Onemax}}
\begin{tabular}{|r|r|rr|rr|}
\hline
\multicolumn{1}{|c|}{Samples} & \multicolumn{1}{c|}{Cutoff} &
 \multicolumn{2}{c|}{Best} & 
 \multicolumn{2}{c|}{Evaluations}  \\ \hline
6000 & 0.7 & --399.9 & (0.3) & 538800 & (31269.15) \\ \hline
6000 & 0.5 & --394.6 & (3.56) & 2835000 & (207000) \\ \hline
3000 & 0.7 & --380.1 & (8.83) & 272400 & (7800) \\ \hline
3000 & 0.5 & --359.2 & (7.15) & 2901000 & (0) \\ \hline
1000 & 0.7 & --339.3 & (9.18) & 72200 & (3124.1) \\ \hline
500 & 0.7 & --319.5 & (4.92) & 32300 & (1661.32) \\ \hline
1000 & 0.5 & --317.3 & (8.96) & 279600 & (27122.68) \\ \hline
500 & 0.5 & --298 & (5.59) & 74550 & (4660.74) \\ \hline
100 & 0.7 & --286.3 & (4.24) & 4870 & (272.21) \\ \hline
100 & 0.5 & --273.9 & (4.87) & 8120 & (622.58) \\ \hline
500 & 0.3 & --269.1 & (7.33) & 2900500 & (0) \\ \hline
1000 & 0.3 & --265.7 & (3.13) & 2901000 & (0) \\ \hline
3000 & 0.3 & --264.8 & (1.83) & 2901000 & (0) \\ \hline
6000 & 0.3 & --264 & (1.55) & 2904000 & (0) \\ \hline
100 & 0.3 & --254.9 & (8.35) & 2900100 & (0) \\ \hline
\end{tabular}
\label{ce-onemax}
\end{table}

\begin{table}
\centering
\caption{\tabletitle{CE}{1D Ising}}
\begin{tabular}{|r|r|rr|rr|}
\hline
\multicolumn{1}{|c|}{Samples} & \multicolumn{1}{c|}{Cutoff} &
 \multicolumn{2}{c|}{Best} & 
 \multicolumn{2}{c|}{Evaluations}  \\ \hline
1000 & 0.7 & --288.4 & (7.94) & 2901000 & (0) \\ \hline
500 & 0.7 & --287.6 & (5.99) & 2613250 & (861750) \\ \hline
500 & 0.5 & --273.2 & (5.31) & 2900500 & (0) \\ \hline
100 & 0.7 & --269.8 & (6.72) & 183420 & (206344.9) \\ \hline
500 & 0.3 & --259.2 & (6.21) & 2900500 & (0) \\ \hline
100 & 0.5 & --258.2 & (8.12) & 1931100 & (1221777.89) \\ \hline
1000 & 0.3 & --251.6 & (2.5) & 2901000 & (0) \\ \hline
3000 & 0.5 & --250.8 & (2.56) & 2901000 & (0) \\ \hline
6000 & 0.5 & --250.2 & (2.27) & 2904000 & (0) \\ \hline
1000 & 0.5 & --250 & (2) & 2901000 & (0) \\ \hline
3000 & 0.7 & --249.6 & (2.65) & 2901000 & (0) \\ \hline
6000 & 0.3 & --249.6 & (1.2) & 2904000 & (0) \\ \hline
3000 & 0.3 & --249.4 & (1.8) & 2901000 & (0) \\ \hline
6000 & 0.7 & --249.2 & (1.6) & 2904000 & (0) \\ \hline
100 & 0.3 & --246.6 & (8.67) & 2900100 & (0) \\ \hline
\end{tabular}
\label{ce-1d-ising}
\end{table}


\begin{table}
\centering
\caption{\tabletitle{CE}{2D Ising}}
\begin{tabular}{|r|r|rr|rr|}
\hline
\multicolumn{1}{|c|}{Samples} & \multicolumn{1}{c|}{Cutoff} &
 \multicolumn{2}{c|}{Best} & 
 \multicolumn{2}{c|}{Evaluations}  \\ \hline
1000 & 0.7 & --533 & (12.53) & 2901000 & (0) \\ \hline
500 & 0.7 & --525 & (14.81) & 2613900 & (859800) \\ \hline
500 & 0.5 & --506.6 & (13.97) & 2900500 & (0) \\ \hline
100 & 0.7 & --501.2 & (9) & 168440 & (176983.23) \\ \hline
500 & 0.3 & --484.2 & (16.04) & 2900500 & (0) \\ \hline
100 & 0.5 & --480.4 & (10.07) & 2191050 & (937454.9) \\ \hline
100 & 0.3 & --473.4 & (9.3) & 2900100 & (0) \\ \hline
3000 & 0.5 & --472.8 & (2.56) & 2901000 & (0) \\ \hline
3000 & 0.7 & --472.6 & (3.23) & 2901000 & (0) \\ \hline
6000 & 0.7 & --472.4 & (3.56) & 2904000 & (0) \\ \hline
6000 & 0.3 & --472.4 & (3.67) & 2904000 & (0) \\ \hline
3000 & 0.3 & --472 & (3.22) & 2901000 & (0) \\ \hline
1000 & 0.5 & --471.4 & (3.9) & 2901000 & (0) \\ \hline
6000 & 0.5 & --471.2 & (2.99) & 2904000 & (0) \\ \hline
1000 & 0.3 & --470.8 & (2.56) & 2901000 & (0) \\ \hline
\end{tabular}
\label{ce-2d-ising}
\end{table}

Tables \ref{ann-onemax}, \ref{ann-1d-ising}, and \ref{ann-2d-ising}
show the results of EDA.
Tables \ref{ce-onemax}, \ref{ce-1d-ising}, and \ref{ce-2d-ising}
show the results of CE.
The values in the first and second columns are
the number of generated samples per sampling and the cutoff rate value, 
respectively.
The third column lists the average cost function value,
 with the standard deviation
in parenthesis,  of the best obtained solutions over ten independent runs.
The forth column lists the number of function evaluations
until the population converges.
The convergence criterion is that the number of function evaluations
is greater than $2.9e6$ or the variance of the cost function values
of the generated samples is less than $1e-20$.

Clearly, the performance of CE is inferior to that of EDA, 
despite the fact that
CE is  basically equivalent to or plausibly better than EDA.
The results show that the populations of CE do not converge well.
The performance of CE is dramatically improved in \ref{chapter-rpm}
by adding a population mechanism.
